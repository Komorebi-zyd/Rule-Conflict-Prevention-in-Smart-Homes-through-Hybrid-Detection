\section{Introduction}

% 智能家居代表了物联网(IoT)的一项重要应用,通过将各类智能设备互联,为追求生活质量的用户提供了前所未有的便利性和舒适度。其智能化的核心驱动力在于自动化规则,通常采用“触发器-条件-动作”(Trigger-Condition-Action, TCA)的模式。通过用户自定义或预配置的自动化规则,智能家居能够感知环境变化并自主执行任务,极大简化了用户的日常操作。触发器定义了规则启动的事件;条件是对触发事件发生后执行动作的先决判断;动作为满足条件时执行的具体操作。例如,一条旨在维持恒温的规则可以表示为:$\langle$ 室内温度低于24\celsius(触发器),空调处于关闭状态(条件),开启空调并设为27\celsius(动作) $\rangle$。


The smart home represents a crucial application domain of the Internet of Things (IoT), providing unprecedented convenience and comfort to users seeking a higher quality of life by interconnecting various smart devices. The core driving force behind its intelligence lies in automation rules, typically adopting the Trigger-Condition-Action (TCA) model. Through user-defined or pre-configured automation rules, the smart home can sense environmental changes and autonomously execute tasks, greatly simplifying users' daily routines. The Trigger defines the event that initiates the rule; the Condition is the prerequisite judgment for executing the action after the trigger event occurs; and the Action is the specific operation performed when the condition is met. For instance, a rule designed to maintain a constant temperature can be expressed as: $\langle$ Indoor temperature is below 24\celsius\ (Trigger), Air conditioner is off (Condition), Turn on the air conditioner and set to 27\celsius\ (Action) $\rangle$.

% 然而,随着家庭中部署的规则数量增加,这种便利性往往伴随着隐患。在复杂的智能家居系统中,多条规则的并发执行可能导致规则间产生意想不到的交互,进而引发被称为“规则冲突”(Rule Conflict)的严重后果。这些冲突不仅会扰乱预期的自动化流程,更可能造成安全威胁或物理损害。如图 \ref{rule_conflict_example} 所示,规则 R1(烟雾触发喷淋)和规则 R2(漏水触发关闭水阀)在独立运行时均符合逻辑。但当它们同时存在时,R1 的执行会触发 R2,导致水阀关闭,从而切断喷淋系统的水源,直接阻碍火灾扑救。这种由合法的单条规则组合而成的系统级异常,是当前智能家居安全面临的一大挑战。
However, as the number of rules deployed in the home increases, this convenience is often accompanied by hidden dangers. In complex smart home systems, the concurrent execution of multiple rules can lead to unexpected rule interactions, subsequently resulting in serious consequences known as Rule Conflicts. These conflicts not only disrupt the intended automation flow but may also cause security threats or physical damage. As shown in Figure~\ref{rule_conflict_example}, Rule R1 (Smoke triggers sprinkler) and Rule R2 (Water leak triggers water valve closure) are individually logical when running independently. Yet, when they coexist, the execution of R1 triggers R2, leading to the closing of the water valve, which in turn cuts off the water supply to the sprinkler system, directly hindering firefighting efforts. This system-level anomaly, arising from the interaction of individually legitimate rules, represents a major challenge facing current smart home security.

\begin{figure}[htbp]
	\centering
	\includegraphics[width=0.4\textwidth]{figure/rule_conflict_example.png}
	\caption{Rule Conflict Example}
	\label{rule_conflict_example}
\end{figure}

% 针对这一问题,学术界已展开了广泛研究\cite{alhanahnah2020scalable,chen2019multi,chi2020cross,ding2021iotsafe,huang2021conflict,li2020diac,xiao2019a3id,nakamura2005feature,igaki2010modeling,ibrhim2020formal,pradeep2021automating,shehata2007using,sun2014conflict,alharithi2019detecting,celik2019iotguard,hamza2022hsas,leelaprute2008detecting,trimananda2020understanding,yagita2015application,yu2022tapinspector,chaki2020fine,chi2023detecting,celik2018soteria,alhanahnah2022iotcom,ding2018safety,hsu2019safechain,wang2019charting}。现有的检测方法主要分为基于安全策略的检测方法和基于交互模式的检测方法两类。前者依赖预定义的安全规则(如“无人在家时,门必须关闭”)来检测违规行为\cite{celik2018soteria,celik2019iotguard,ding2021iotsafe};后者通过识别特定的规则交互模式(如循环触发)来预警潜在风险\cite{chi2020cross,chi2023detecting,celik2019iotguard,alhanahnah2022iotcom}。在冲突缓解(Conflict Mitigation)方面,主流手段包括静态的规则重配置(Rule Reconfiguration)、强制执行通用策略或针对特定场景的定制化处理。
In response to this issue, extensive research has been conducted in academia \cite{alhanahnah2020scalable,chen2019multi,chi2020cross,ding2021iotsafe,huang2021conflict,li2020diac,xiao2019a3id,nakamura2005feature,igaki2010modeling,ibrhim2020formal,pradeep2021automating,shehata2007using,sun2014conflict,alharithi2019detecting,celik2019iotguard,hamza2022hsas,leelaprute2008detecting,trimananda2020understanding,yagita2015application,yu2022tapinspector,chaki2020fine,chi2023detecting,celik2018soteria,alhanahnah2022iotcom,ding2018safety,hsu2019safechain,wang2019charting}. Existing detection methods are primarily categorized into Security Policy-based methods and Interaction Pattern-based methods. The former relies on predefined security rules (e.g., “The door must be closed when no one is home”) to detect violations \cite{celik2018soteria,celik2019iotguard,ding2021iotsafe}; the latter identifies specific rule interaction patterns (e.g., cyclic triggering) to pre-warn potential risks \cite{chi2020cross,chi2023detecting,celik2019iotguard,alhanahnah2022iotcom}. Regarding Conflict Mitigation, mainstream approaches include static Rule Reconfiguration, enforcing generic policies, or customized handling strategies for specific scenarios.

% 尽管现有方案取得了一定进展,但仍面临严峻挑战,主要体现在检测的准确性与处理的灵活性之间难以平衡。
% 首先,在冲突检测方面:1)通用策略缺乏个性化。智能家居环境高度异构且用户偏好主观性强。相同的规则交互(如烟雾触发喷淋)在不同家庭(如有明火壁炉的家庭 vs 普通家庭)可能有截然不同的安全含义。一刀切的策略导致大量误报或漏报。2)物理通道感知的局限性。许多规则通过物理环境(温度、光照)间接交互。忽略区域特征(如卧室与客厅的热隔离)会导致对交互路径的错误判断。例如,SafeChain\cite{hsu2019safechain}虽然声称是动态的,但主要基于静态模型更新,缺乏对运行时物理状态的实时感知;而IoTGuard\cite{celik2019iotguard}忽略了物理通道的复杂性,仅关注应用层面的动作冲突。
Despite the progress achieved by existing solutions, severe challenges remain, mainly reflected in the difficulty of balancing detection accuracy and handling flexibility.
Firstly, concerning rule conflict detection: 1) Generic policies lack personalization. The smart home environment is highly heterogeneous, and user preferences are intensely subjective. The same rule interaction (e.g., smoke triggering the sprinkler) might have drastically different security implications in different homes (e.g., a home with a live-fire fireplace vs. a typical home). A one-size-fits-all strategy leads to a high rate of false positives or false negatives. 2) Limitations in physical channel awareness. Many rules interact indirectly through the physical environment (e.g., temperature, light). Ignoring zone characteristics (e.g., thermal isolation between the bedroom and the living room) can lead to incorrect judgments about the interaction path. For example, SafeChain\cite{hsu2019safechain}, while claiming to be dynamic, is primarily based on static model updates, lacking real-time perception of runtime physical states; while IoTGuard\cite{celik2019iotguard} overlooks the complexity of the physical channel, focusing only on application-level action conflicts.

% 其次,在冲突缓解方面:1)静态修复的脆弱性。简单的规则重配置(如删除某条规则)往往破坏了用户预期的自动化功能,甚至引入新问题。2)动态拦截的粗粒度。现有的动态系统如IoTSafe\cite{ding2021iotsafe}虽然能进行预防性拦截,但往往采用通用的阻断策略,缺乏针对不同冲突类型的精细化、定制化处理能力。IoTMediator\cite{chi2023detecting}虽然尝试了定制化处理规则冲突,但是其本质上是针对不同规则冲突模式设置了不同的冲突环节模板,并且需要用户主动去选择,而缺少一套自动化的执行方法,且该工作在区分“良性规则交互”与“恶性规则冲突”方面仍有不足。
Secondly, concerning conflict mitigation: 1) Fragility of static fixes. Simple rule reconfiguration (e.g., deleting a rule) often breaks the user's intended automation functionality, or even introduces new problems. 2) Coarse granularity of dynamic interception. Existing dynamic systems like IoTSafe\cite{ding2021iotsafe} can perform preventative interception but often employ a generic blocking strategy, lacking fine-grained, customized handling strategy capabilities for different conflict types. Although IoTMediator\cite{chi2023detecting} attempted customized handling of rule conflicts, it essentially sets up different conflict stage templates for different rule conflict patterns and requires active user selection, lacking an automated execution method. Furthermore, this work still falls short in distinguishing between "benign rule interaction" and "malicious rule conflict."

% 针对上述问题,本文基于以下核心观察提出解决方案:(1)规则交互不仅存在于逻辑层,更通过物理通道在不同区域间传播,必须建立包含区域特征的增强模型;(2)规则冲突的判定具有主观性,必须区分客观的“规则交互”与主观有害的“规则冲突”;(3)智能家居的运行时特性允许我们在冲突发生的毫秒级窗口内进行干预。基于此,我们提出了一种混合框架,结合了静态分析的全面性与动态监测的实时性。我们首先通过形式化验证构建“规则交互依赖图”(RIDG),利用用户定义的实体安全配置从中提取“规则冲突依赖图”(RCDG),并为每个冲突生成定制化策略。在运行时,系统利用断言验证机制实时监测冲突路径的激活情况,并由控制器执行精准拦截。
This paper proposes a solution based on the following core observations: (1) Rule interaction exists not only at the logical layer but also propagates through the physical channel across different zones; thus, an enhanced model incorporating zone characteristics must be established. (2) The judgment of rule conflict is subjective; we must distinguish between objective "rule interaction" and subjectively harmful "rule conflict." (3) The runtime characteristics of the smart home allow us to intervene within the millisecond window before a conflict manifests. Based on these, we propose a hybrid framework combining the comprehensiveness of static analysis with the real-time nature of dynamic monitoring. We first construct a Rule Interaction Dependency Graph (RIDG) through formal verification, extract the Rule Conflict Dependency Graph (RCDG) using user-defined entity security configurations, and generate a customized handling strategy for each conflict. At runtime, the system utilizes an assertion verification mechanism to monitor the real-time activation of conflict paths, and a controller performs precise interception.

% 综上所述,本文的主要贡献如下:
In summary, our main contributions are as follows:

% \item 我们提出了一种基于物理通道感知和区域特征的规则冲突分类新方法。通过引入TCAE模型和RIDG图结构,我们涵盖了所有类型的规则交互,并清晰界定了“交互”与“冲突”的边界,有效解决了物理通道引起的漏报和误报问题。

% \item 我们设计并实现了一种混合式冲突管理框架。静态阶段结合实体安全配置进行精确的冲突预筛选;动态阶段利用高效的断言验证机制,在运行时实时监测并确认冲突的发生,实现了低开销的精准检测。

% \item 我们实现了自动化且个性化的冲突缓解机制。系统能够根据冲突类型和安全配置自动生成定制化处理策略,并在冲突发生的临界时刻进行预防性拦截,既保障了安全,又最大程度维持了系统的可用性。\\

\begin{itemize}
	\item We propose a novel method for rule conflict classification based on physical channel awareness and zone characteristics. By introducing the TCAE model and the Rule Interaction Dependency Graph (RIDG) structure, we cover all types of rule interaction and clearly delineate the boundary between "interaction" and "conflict," effectively solving the problems of false negatives and false positives caused by the physical channel.
	
	\item We design and implement a hybrid conflict management framework. The static phase performs precise conflict pre-screening combined with entity security configurations; the dynamic phase utilizes an efficient assertion verification mechanism to monitor and confirm the occurrence of conflicts in real-time at runtime, achieving low-overhead, precise detection.
	
	\item We implement an automated and personalized conflict mitigation mechanism. The system can automatically generate customized handling strategies based on the conflict type and security configuration, and perform preventative interception at the critical moment of conflict occurrence, thus ensuring safety while maximizing system usability.
\end{itemize}