\section{Motivating Example}

\begin{figure}[htbp]
	\centering
	\includegraphics[width=0.4\textwidth]{figure/motivated_example.png}
	\caption{Motivating Example}
	\label{motivated_example}
\end{figure}

% 本节提供了一个具体的例子,以说明规则冲突的特征,并分析现有冲突检测和缓解方法的局限性。图\ref{motivated_example}显示了一个智能家居平面图,包括七个区域:室外(顶部中心)、门廊(中心)、客厅(底部中心)、厨房(右下)、浴室(右下)、卧室(左中)和温室(左下)。其中的\circled{1}是加热器,\circled{2}是卧室窗户,\circled{3}是加热器,\circled{4}是greenhouse的天窗。
This section provides a concrete example to illustrate the characteristics of rule conflicts and analyze the limitations of existing conflict detection and mitigation methods. Figure \ref{motivated_example} shows a smart home floor plan including seven zones: outdoor (top center), porch (center), living room (bottom center), kitchen (bottom right), bathroom (bottom right), bedroom (left center), and greenhouse (bottom left). \circled{1} is a heater, \circled{2} is a bedroom window, \circled{3} is a heater, and \circled{4} is a greenhouse skylight.

% 考虑在温室中设置以下两条规则:规则 1 (R1):当检测到日落时 (触发器),开启加热器 (动作);规则 2 (R2): 当温度达到 30 摄氏度时 (触发器),打开窗户并关闭加热器 (动作)。如图 \ref{motivated_example} 左下角所示,日落时,规则 R1 自动执行,为温室供暖。随着温度持续升高,可能超过 30 摄氏度,触发规则 R2 的执行,导致窗户打开且加热器关闭。在温室中,这种规则交互通常符合用户预期,且不会造成安全问题。然而,若将这两条规则应用于卧室,情况则截然不同。规则 R1 通过提高卧室温度触发规则 R2,进而打开窗户。卧室的窗户通常不是自动天窗,且在用户不知情的情况下打开窗户会带来安全隐患。攻击者甚至可能利用这些规则,创建恶意执行路径,从而控制窗户、门锁等安全敏感设备。
	Consider the following two rules set in the greenhouse: Rule 1 (R1): When sunset is detected (trigger), turn on the heater (action); Rule 2 (R2): When the temperature reaches 30\celsius (trigger), open the window and turn off the heater (action). As shown in the bottom left corner of Fig. \ref{motivated_example}, at sunset, rule R1 is automatically executed, providing heat to the greenhouse. As the temperature continues to rise, it may exceed 30\celsius, triggering the execution of rule R2, which causes the window to open and the heater to turn off. In a greenhouse, this rule interaction is usually in line with user expectations and does not cause security problems. However, if these two rules are applied to the bedroom, the situation is very different. Rule R1 triggers the execution of rule R2 by raising the bedroom temperature, which in turn opens the windows. Bedroom windows are usually not automatic skylights, and opening the windows without the user's knowledge can create security risks. Attackers may even exploit these rules to create malicious execution paths to control security-sensitive devices such as windows and door locks.

% 上述示例揭示了规则交互的复杂性。规则间的相互影响并非仅限于直接作用 (例如,两条同时执行的规则,一条设置为制热模式,另一条设置为制冷模式),也可能通过温度、湿度、亮度、声音等间接途径产生。因此,在检测规则冲突时,必须考虑来自其他渠道的潜在冲突。此外,规则冲突具有地域性和用户主观性。相同的规则交互在温室中可能被认为是正常行为,而在卧室中则可能被视为冲突。用户主观性体现在对同一组规则交互的不同解读:一些用户可能认为卧室空调的除湿模式和加湿器交替工作是为了维持空气湿度,属于正常交互,而另一些用户则可能认为这是一种资源浪费。再者,智能家居系统中对侧面通道的检测不应仅限于温度、湿度等参数,还应具备地域属性。例如,门廊的灯光可能影响厨房和客厅的亮度,但通常不会影响卧室。因此,同时执行的两条与亮度相关的规则可能不会实际发生交互。
The above example reveals the complexity of rule interactions. The interaction between rules is not limited to direct effects (e.g., two rules executing simultaneously, one set to heating mode and the other to cooling mode), but can also be generated through indirect channels such as temperature, humidity, brightness, and sound. Therefore, when detecting rule conflicts, it is necessary to consider potential conflicts from other channels. In addition, rule conflicts have regionality and user subjectivity. The same rule interaction may be considered normal behavior in a greenhouse, but may be considered a conflict in a bedroom. User subjectivity is reflected in different interpretations of the same set of rule interactions: Some users may consider the alternating operation of the dehumidification mode and the humidifier in the bedroom to maintain air humidity as a normal interaction, while others may consider it a waste of resources. Furthermore, the detection of side channels in smart home systems should not be limited to parameters such as temperature and humidity, but should also have regional attributes. For example, porch lights may affect the brightness of the kitchen and living room, but usually do not affect the bedroom. Therefore, two brightness-related rules executed at the same time may not actually interact.

% 若采用基于安全策略的方法进行冲突检测,用户需投入大量精力,因为不同家庭的户型、设备配置等差异会导致安全策略的定制需求各不相同。这不仅需要专业人员的参与,而且定制的安全策略很可能无法涵盖所有可能的冲突场景。另一方面,若仅根据规则交互模式进行判断,虽然能检测到卧室中发生的冲突,但也会将温室中的规则交互误判为冲突。随着规则数量的增加,误报数量会急剧上升,用户需要逐一核实,从而显著增加用户负担。
If a security policy-based approach is used for conflict detection, users need to invest a lot of effort, because the customization needs of security policies will vary from home to home due to differences in housing types, equipment configurations, etc. This not only requires the participation of professionals, but the customized security policies are likely to fail to cover all possible conflict scenarios. On the other hand, if the judgment is based solely on the rule interaction mode, the conflicts occurring in the bedroom can be detected, but the rule interaction in the greenhouse will be misjudged as a conflict. As the number of rules increases, the number of false positives will increase sharply, and users need to verify them one by one, thus significantly increasing the user burden.

% 若采取更改规则配置的方法,可能会避免规则冲突,但是更多的情况是无法实现用户期望的自动化功能,甚至引入其他规则冲突。若采取基于安全策略的方法避免规则冲突,在当前智能家居系统中比较容易实现,但是该方法很难推广到环境各异的不同用户家庭。
If the method of changing rule configurations is adopted, rule conflicts may be avoided, but in most cases, the expected automation functions cannot be achieved, and even other rule conflicts may be introduced. If the method based on security policies is adopted to avoid rule conflicts, it is relatively easy to implement in the current smart home system, but this method is difficult to promote to different users' home with different environments.

% 由此我们总结出以下挑战并给出解决思路。
From this, we summarize the following challenges and provide solutions.

%智能家居系统多样性带来的规则冲突检测方面的困难:不同用户的智能家居系统具有差异性,家庭中布局、智能家居设备多种多样,特定的安全策略虽然能够有效避免已经发现的规则冲突,但是其泛化能力弱,难以移植到不同家庭环境,且不能覆盖所有规则冲突,不能广泛使用,采用检测规则交互的判定方法容易产生较多误报,从而导致用户工作量较大,设计一个泛化能力强且检测效果优异的冲突检测方法是一个挑战

%我们采用逐步提纯的思路寻找规则冲突:首先结合用户家庭环境特征对规则进行重新建模,用户只需要根据指导提供与规则相关的区域信息即可快速完成;然后进行形式化分析,检测出所有可能的规则交互;接下来将根据用户提交的实体安全信息,自动化检测出违反实体安全信息的规则交互,视为规则冲突。

%用户主观态度对冲突判定的影响带来的自动化冲突检测方面的困难:一次规则交互是否是规则冲突取决于用户的主观感受,或者称为个人偏好,如何将用户偏好结合到规则冲突的判断中,且保证自动化程度高,便于非专业用户操作是一项挑战

%在规则冲突检测的过程中使用用户友好界面,指导用户提供需要的信息,如自动化检测出违反实体安全信息的规则交互被视为规则冲突后,将会展示出详细的冲突信息,用户可以根据个人偏好选择冲突处理策略,用户也可以检查所有的规则交互信息(自然语言格式),并指定规则交互处理策略,避免用户不期望的规则交互发生。

%智能家居系统多样性带来的规则冲突处理方面的困难:特定的安全策略针对特定的家庭,相同的安全策略很难推广到不同家庭,且该方法只能用于防止已经发生的规则冲突,它与重新配置规则都具有相同的缺陷:用户工作量大、需要具有相关专业知识的人员指导、难以大范围推广,设计一个有效自动化程度高、便于非专业用户实现的规则冲突处理策略是一个挑战。

%我们设计了针对不同规则冲突的定制化处理方案。针对不同类型的规则冲突,设计多种处理策略模板,并根据具体的冲突信息进行模板填充,每个规则冲突都能够得到针对性处理策略选项,并根据规则冲突的特征选择相应的处理策略,并且在规则冲突发生前拦截系统默认操作,执行冲突处理策略,从而避免规则冲突的发生。


\begin{itemize}
	\item \textbf{C1}: The difficulties in rule conflict detection caused by the diversity of smart home systems: Smart home systems vary among different users, with diverse layouts and smart home devices in households. Although specific security policies can effectively avoid detected rule conflicts, their generalization ability is weak, making them difficult to transfer to different home environments and failing to cover all rule conflicts. They cannot be widely used. The method of determining rule interaction is prone to generating many false positives, resulting in a large workload for users. Designing a conflict detection method with strong generalization ability and excellent detection effect is a challenge.
	\item \textbf{S1}:We adopt a step-by-step refinement approach to identify rule conflicts: first, we remodel the rules by incorporating the characteristics of the user's home environment. Users only need to provide region information related to the rules according to the guidance to quickly complete the setup. Then, we perform formal analysis to detect all possible rule interactions. Next, based on the entity security information submitted by the user, we automatically detect rule interactions that violate entity security information, which are considered rule conflicts.
	
	\item \textbf{C2}: The difficulty of automated conflict detection caused by the influence of users' subjective attitudes on conflict determination: Whether a rule interaction is a rule conflict depends on the user's subjective feelings, or personal preferences. How to incorporate user preferences into the judgment of rule conflicts while ensuring a high degree of automation and ease of operation for non-professional users is a challenge.
	\item \textbf{S2}: In the rule conflict detection process, a user-friendly interface is used to guide users in providing the necessary information. For example, after automatically detecting rule interactions that violate entity security information as rule conflicts, detailed conflict information will be displayed. Users can choose conflict resolution strategies according to their personal preferences. Users can also check all rule interaction information (in natural language format) and specify rule interaction handling strategies to avoid unwanted rule interactions.
	
	\item \textbf{C3}: The difficulty in dealing with rule conflicts caused by the diversity of smart home systems: Specific security policies are designed for specific homes, and the same security policies are difficult to promote to different homes. Moreover, this method can only be used to prevent rule conflicts that have already occurred. It shares the same drawbacks as reconfiguring rules: high user workload, the need for guidance from professionals with relevant expertise, and difficulty in large-scale promotion. Designing a highly automated rule conflict resolution strategy that is easy for non-professional users to implement is a challenge.
	\item \textbf{S3}: We have designed customized processing schemes for different rule conflicts. For different types of rule conflicts, we design multiple processing strategy templates and fill them in according to specific conflict information. Each rule conflict can obtain targeted processing strategy options, and the corresponding processing strategy is selected according to the characteristics of the rule conflict. Furthermore, the system default operation is intercepted before the rule conflict occurs, and the conflict processing strategy is executed, thereby avoiding the occurrence of rule conflicts.
\end{itemize}