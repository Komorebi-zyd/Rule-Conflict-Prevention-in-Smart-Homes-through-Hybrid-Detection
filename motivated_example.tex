\section{Motivating Example and Problem Analysis}

\begin{figure}[htbp]
	\centering
	\includegraphics[width=0.4\textwidth]{figure/motivated_example.png}
	\caption{Motivating Example}
	\label{motivated_example}
\end{figure}

% 本节提供了一个具体的例子来说明规则冲突的表现形式,并分析了现有冲突检测和缓解方法的局限性。图\ref{motivated_example} 展示了一个智能家居系统的两个区域的平面图:卧室(左侧)和温室(右侧)。\circled{1}是卧室的取暖器,\circled{2}是卧室的窗户,\circled{3}是温室的加热器,\circled{4}是温室的天窗。
This section provides a concrete example to illustrate the manifestation of rule conflicts and analyzes the limitations of existing conflict detection and mitigation methods. \textbf{Fig}ure~\ref{motivated_example} shows the floor plan of two zones within a smart home system: the bedroom (left) and the greenhouse (right). \circled{1} is the heater in the bedroom, \circled{2} is the window in the bedroom, \circled{3} is the heater in the greenhouse, and \circled{4} is the skylight in the greenhouse.

% 考虑在温室中设置的以下两条自动化规则:规则 1 (R1):当检测到日落(触发器)时,打开加热器(动作);规则 2 (R2):当温度达到 30 摄氏度时(触发器),打开窗户并关闭加热器(动作)。
Consider the following two automation rules set in the greenhouse: Rule 1 (R1): When sunset is detected (trigger), turn on the heater (action); Rule 2 (R2): When the temperature reaches 30\celsius\ (trigger), open the window and turn off the heater (action).

\subsection{Challenge: Diversity of Conflict Influencing Factors}

% 智能家居环境中规则冲突的判定具有高度的复杂性。与传统的软件逻辑错误不同,自动化规则之间的交互是否构成“冲突”,不仅取决于规则本身的逻辑结构,还受到多种外部因素的显著影响。现有的研究往往难以全面采集这些多样化的影响因素,导致冲突检测出现大量误报或漏报。具体而言,这些挑战主要体现在以下三个方面:
Defining rule conflicts in the smart home environment is highly complex. Unlike traditional software logic errors, whether the interaction between automation rules constitutes a "conflict" depends not only on the logical structure of the rules themselves but is also significantly influenced by various external factors. Existing researches are hard to capture these diverse influencing factors comprehensively, leading to high rates of false positives or false negatives in conflict detection. Specifically, these challenges are manifested in the following three aspects:

% 1. 用户偏好难以被采集(主观性)。规则冲突的定义在很大程度上取决于用户的主观意图。同样的规则交互模式对某些用户来说可能是预期的功能,而对其他用户则可能是严重的威胁。如图 \ref{motivated_example} 所示,当 R1 和 R2 部署在温室中时,R1 的加热导致温度升高从而触发 R2 打开天窗,这通常是用户为了保持恒温而故意设计的闭环控制。然而,如果相同的规则应用于卧室(例如 R1 对应卧室加热器,R2 对应卧室窗户),在夜间自动打开窗户会带来严重的安全隐患,这种“意外后果”则构成了规则冲突。此外,对于特殊群体,偏好的差异更为关键。例如,光敏性癫痫患者对灯光变化极为敏感,如果两条规则的交互导致灯光频繁闪烁(死锁循环),这对普通用户可能只是干扰,但对该类患者则是严重的安全威胁。现有方法难以捕捉这种细粒度的、主观的用户偏好。
\textbf{1. Difficulty in Capturing User Preferences (Subjectivity).} The definition of a rule conflict largely depends on the subjective intent of the user. The same rule interaction pattern may be an expected feature for some users while posing a serious threat to others. As shown in Figure~\ref{motivated_example}, when R1 and R2 are deployed in the greenhouse, R1's heating causes the temperature to rise, thereby triggering R2 to open the skylight. This is typically a closed-loop control intentionally designed by the user to maintain a constant temperature. However, if the same rules are applied to the bedroom (e.g., R1 controls the bedroom heater, and R2 controls the bedroom window), opening the window automatically at night poses a significant security risk, and this "unintended consequence" constitutes a rule conflict. Furthermore, for specific groups, differences in preferences are even more critical. For instance, patients with photosensitive epilepsy are extremely sensitive to light changes. If the interaction of two rules causes lights to flicker frequently (a deadlock loop), this may be a mere nuisance to ordinary users but a severe safety threat to such patients. Existing methods are hard to capture such fine-grained and subjective user preferences.

% 2. 智能家居系统的多样性。自动化规则往往通过物理环境(即物理通道)产生间接交互,而这种交互高度依赖于设备的物理位置和区域特征。例如,R1 打开加热器会导致温度升高。如果 R2 的触发条件是“当客厅温度高于 30度时打开窗户”,而 R1 的加热器位于卧室,由于卧室和客厅的热隔离,R1 的执行实际上不会触发 R2。然而,如果忽略了这种区域(Zone)的物理隔离特性,仅凭“加热器导致温度升高”这一逻辑推断,检测系统就会发出误报。目前的许多方法缺乏对这种复杂的物理环境和区域特征的精确建模能力。
\textbf{2. Smart Home System Diversity.} Automation rules often interact indirectly through the physical environment (i.e., physical channels), and this interaction is highly dependent on the physical location of devices and zonal characteristics. For example, R1 turning on the heater causes the temperature to rise. If the trigger condition of R2 is "Open the window when the living room temperature is above 30\celsius," and the heater controlled by R1 is located in the bedroom, the execution of R1 will not actually trigger R2 due to the thermal isolation between the bedroom and the living room. However, if this physical isolation characteristic of the "Zone" is ignored and reliance is placed solely on the logical inference that "heater causes temperature rise," the detection system will generate false positives. Many current methods lack the capability to precisely model such complex physical environments and regional characteristics.

% 3. 实体安全配置难以被采集。判断一个规则交互是否有害,通常需要依据安全策略。以往的方法依赖通用的安全策略(例如“当无人在家时,门必须保持锁闭”)。然而,智能家居环境是个性化的,通用的安全策略难以覆盖所有场景。对于普通用户而言,手动为家中的每一个设备定义详尽的安全状态(Entity Safety Configuration)不仅门槛过高,而且极其繁琐。缺乏准确的、符合特定家庭环境的实体安全配置,使得系统难以区分哪些设备状态变更是允许的,哪些是危险的规则冲突。
\textbf{3. Difficulty in Capturing Entity Safety Configurations.} Determining whether a rule interaction is harmful typically up to security policies. Previous methods depended on generic security policies (e.g., "The door must remain locked when no one is at home"). However, the smart home environment is personalized, and generic security policies cannot cover all scenarios. For ordinary users, manually defining a detailed Entity Safety Configuration for every device in the home is not only a high barrier but also extremely cumbersome. The lack of accurate entity safety configurations tailored to a specific home environment makes it difficult for the system to distinguish between permissible device state changes and dangerous rule conflicts.

\subsection{Our Solution: User-Centric Configuration and Interaction}

% 为了应对上述冲突影响因素多样性的挑战,本文提出了一套结合精细化配置与用户友好交互的解决方案。
To address the challenge of the diversity of conflict influencing factors, this paper proposes a solution that combines refined configuration with user-friendly interaction.

% 首先,我们引入了基于配置文件的环境与安全建模方法。为了解决系统多样性问题,我们设计了包含区域(Zone)、通道(Channel)和趋势(Trend)的物理通道配置方法,允许系统准确理解“卧室加热器”只会影响“卧室温度”。同时,为了解决安全策略缺失的问题,我们引入了实体安全配置(Entity Safety Configuration),明确定义关键设备在冲突发生时的首选安全状态(例如,窗户倾向于“关闭”)。
First, we introduce an environment and safety modeling approach based on configuration files. To address the issue of system diversity, we design a Physical Channel Configuration method that includes Zone, Channel, and Trend. This allows the system to accurately understand that a "bedroom heater" only affects the "bedroom temperature". Simultaneously, to address the lack of security policies, we introduce the Entity Safety Configuration, which explicitly defines the preferred safety state of critical devices when a conflict occurs (e.g., a window tends to be "Closed").

% 其次,为了解决用户偏好难以采集及配置复杂的问题,我们采用了一种用户友好的交互式方法。我们根据规则模型与交互类型的特定展示形式设计翻译模板,将晦涩难懂的规则代码和潜在的交互关系转化为易于理解的自然语言描述。例如,系统不再展示复杂的逻辑符号,而是向用户呈现:“检测到规则 R1(如果日落,打开加热器)的执行可能通过温度通道意外触发规则 R2”。这使得普通用户能够轻松理解规则交互的后果,并基于个人偏好(如是否接受温室的自动开窗)进行二次确认。通过这种“人在回路”(User-in-the-loop)的方式,我们有效地解决了主观性带来的定义模糊问题,确保了冲突检测的准确性和个性化。
Second, to address the difficulty in capturing user preferences and the complexity of cofiguration, we adopt a user-friendly interactive approach. We utilize natural language processing techniques to translate obscure rule codes and potential interaction relationships into easily understandable natural language descriptions. For instance, instead of displaying complex logical symbols, the system presents to the user: "It is detected that the execution of Rule R1 (If sunset, turn on heater) may unexpectedly trigger Rule R2 via the Temperature channel." This enables ordinary users to easily comprehend the consequences of rule interactions and perform secondary confirmation based on personal preferences (e.g., whether to accept automatic window opening in the greenhouse). Through this "User-in-the-loop" approach, we effectively resolve the ambiguity caused by subjectivity, ensuring the accuracy and personalization of conflict detection.