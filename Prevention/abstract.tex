\begin{abstract}
% 随着物联网的快速发展,智能家居日益普及。智能家居系统在很大程度上依赖于自动化规则,即触发-条件-动作(即触发-条件-动作,简称TCA)模型。然而,然而多个规则的相互联动可能会导致用户预期之外的结果,即规则冲突,这将引发不可预期的安全风险。现有的规则冲突检测方法通常存在检测效率低、无法实时监测系统中的规则冲突以及漏报和误报等问题。而相关的冲突缓解方法往往局限于修改规则配置或遵守既定的安全策略(如操作黑名单、白名单),应对策略有限且缺乏个性化的保护。

% 本文提出了一种新颖的框架,结合离线的静态规则分析、运行时的动态监测以及定制化的冲突处理,以实现对智能家居系统规则交互的实时检测与冲突缓解。在静态规则分析方面:我们设计了新的规则模型,并通过分析规则交互模式生成了规则交互依赖图(其中自动化规则作为节点,交互模式作为边)。接着,我们采用基于安全等级的形式化验证方法识别出潜在的规则冲突,并将这些冲突关系从交互图中提炼出来,构成规则冲突依赖图(作为前者的子图)。最后,我们使用启发式方法生成定制化的规则冲突处理策略。在动态监测阶段:系统实时捕获规则事件的发生。对于任何被触发的规则,我们从规则冲突依赖图中识别出其对应的节点,并检查其邻接节点(即存在直接冲突依赖关系的规则)。基于这些潜在的冲突关系,我们应用断言验证机制:验证从当前触发规则到邻接规则的依赖边是否在当前系统状态下被激活。若断言验证成立,表明冲突的必要条件已满足,系统则立即中断默认操作,转而执行预设的冲突缓解策略,从而在冲突发生之前有效地防止冲突。我们使用包含多个区域的17条规则的模拟Home Assistant环境实施并评估了我们的系统。结果表明,与现有工具相比,本系统成功识别了所有手动触发的冲突,包括其他工具遗漏的物理通道引起的冲突,同时显著降低了误报率。本系统能够自动生成可行的、定制的冲突解决策略,并表现出高效的性能:即使对于大型规则集,静态分析也能在几秒钟内完成,而动态开销保持在毫秒级别。我们的方法提供了一种全面、有效和个性化的解决方案来缓解规则冲突。

This paper proposes a novel framework that integrates offline static rule analysis, runtime dynamic monitoring, and a customized handling strategy to achieve real-time dynamic monitoring and conflict mitigation of rule conflicts in the smart home system. In the static analysis phase, we first design a novel rule model, and by analyzing rule interaction patterns, we generate a Rule Interaction Dependency Graph (where automation rules are nodes and interaction patterns are edges). We then employ a security-level-based formal verification method to identify candidate rule conflicts and extract these conflict relations to form a Rule Conflict Dependency Graph (a subgraph of the former). Finally, we utilize a heuristic approach to generate customized handling strategies for these conflicts. For the runtime dynamic monitoring component, the system continuously captures rule events. For any triggered rule, we identify its corresponding node in the Rule Conflict Dependency Graph and check its neighboring nodes (i.e., rules with direct conflict dependency). Based on these potential conflict relations, we apply an assertion verification mechanism to determine whether the dependency edge from the current triggered rule to a neighbor rule is activated under the current system state. If the assertion verification holds, confirming that the necessary condition for a conflict is met, the system immediately intercepts the default rule actions and executes the pre-generated customized handling strategy. This mechanism effectively prevents conflicts before they manifest. We implement and evaluate our system using a simulated Home Assistant environment containing 17 rules across multiple zones. The results demonstrate that our framework successfully identifies all manually verified conflicts in our dataset, including those caused by the physical channels that other tools often overlook, while significantly reducing the false-positive rate. Our system automatically generates feasible, customized conflict handling strategies and exhibits high performance: the static analysis component finishes within seconds, even for large rule sets, and the dynamic overhead remains at the millisecond level. Our approach provides a comprehensive, effective, and personalized solution for the mitigation of rule conflicts.
\end{abstract}

